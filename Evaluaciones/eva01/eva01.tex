\documentclass[]{article}
\usepackage{polyglossia}
\usepackage{fancyhdr}
\usepackage{enumerate}
\usepackage{framed}
\usepackage{listings}
\usepackage{amsfonts}
\usepackage[usenames,dvipsnames,svgnames,table]{xcolor}
\usepackage{minted}
\usepackage{hyperref}
\usepackage{graphicx}
\usepackage{gensymb}
\usepackage{longtable}
\newcommand{\tabitem}{~~\llap{\textbullet}~~}

\setmainlanguage{spanish}
\definecolor{mygray}{RGB}{248,249,250}

\title{Evaluación 1 \protect\\ Taller de Integración de Software}
\author{Anggelo Urso G. \\ anggelo.urso@inacapmail.cl}
\date{\today}

\pagestyle{fancy}
\fancyhf{}
\lhead{Taller de Integración de Software}
\rhead{\thepage}
\lfoot{AUG /\LaTeX}

\hypersetup{
    colorlinks=true,
    linkcolor=blue,
    filecolor=magenta,      
    urlcolor=cyan,
}

\begin{document}
\thispagestyle{empty}
\maketitle

\section{Introducción}
Esta asignatura se basa en el desarrollo de un proyecto de software, el cual se  elaborará durante el semestre y que les permitirá aplicar los conocimientos 
adquiridos y el desarrollo de las competencias hasta este momento de su carrera.

El proyecto será realizado en tres etapas las cuales son:\\

\begin{enumerate}
    \item[Etapa 1] Recopilación, Análisis y Gestión de Requerimientos
    \item[Etapa 2] Desarrollo de Aplicación
    \item[Etapa 3] Presentación y defensa del proyecto.
\end{enumerate} 

En esta primera etapa, tienen que definir, analizar y diseñar el proyecto que desarrollarán durante el semestre. Para esto deben buscar un cliente real 
que les presente una problemática a resolver o generar una propuesta de valor sobre una problemática que el grupo observe. Deberán realizar la captura, 
análisis de requerimientos y evaluación de la factibilidad, para posteriormente generar el diseño de la solución propuesta.

La problemática por considerar debe contemplar un desarrollo web o una aplicación móvil.

Durante el desarrollo de la actividad el académico, les solicitará entregas previas que, en base a las observaciones obtenidas de parte del académico, les 
permitan mejorar o realizar cambios según corresponda. Estas entregas incorporarán la información necesaria que permitirá modelar correctamente el sistema 
y su posterior desarrollo. (Requerimientos, Casos de Uso, Diagramas de Secuencia, Diagramas de Clase, etc.).

\subsection{Criterios de evaluación}

Las actividades consideran los siguientes criterios de evaluación:

\begin{enumerate}
    \item Considerando la especificación de requerimientos del cliente.
    \item Incluye el prototipo que responda a las necesidades de la solución.
    \item Considerando análisis de factibilidad.
    \item Considerando diagramas de clase, casos de uso, secuencia y de procesos, que representen las necesidades de la aplicación.
    \item Considerando el diseño del modelo lógico y físico de datos en base al requerimiento.
    \item Aplicando metodologías ágiles o tradicionales según la naturaleza del proyecto.
    \item Identificando el propio punto de vista.
    \item Utilizando información de fuentes establecidas.
\end{enumerate}

\section{Instrucciones}

\begin{itemize}
    \item La actividad está contemplada para ser desarrollada en forma grupal (3 integrantes) con evaluación individual. Puede ser desarrollada en parejas o 
    de forma individual, sin embargo deberán cumplir igualmente con todos los hitos de la asignatura.
    \item El grupo deberá definir un representante (jefe de proyecto), quien será el interlocutor válido entre el grupo y el académico.
    \item Deberán indicar, justificar y explicar cómo usaran una metodología de desarrollo, a utilizar durante todo el proyecto en base a su naturaleza, la que 
    puede ser una propuesta de solución planteada a un tercero, o la solución a un problema detectado por el grupo.
    \item Deberán rescatar los requerimientos a través de un estudio previo, ya sea con \textit{Stakeholders} dedicados a la propuesta o con aquellos que ustedes
    identifiquen como \textit{Stakeholders} de la solución.
    \item La propuesta deberá considerar requerimientos funcionales, de usuario y no funcionales de todo el proceso.
    \item Los requerimientos deberán encontrarse priorizados, considerando que el desarrollo del proyecto es un MVP (\textit{Minimum Viable Product}) deberán indicar 
    cuales de los requerimientos serán abordados en la presente entrega y cuáles serán abordados o desarrollados en futuras entregas del producto.
    \item Se deberán entregar los casos de uso que dan cuenta de las funcionalidades del sistema, considerando diagramas de alto nivel, nivel detallado y una descripción
    exhaustiva (véase sección \textbf{Actividades}).
\end{itemize}

\section{Actividades}

La etapa 1 del desarrollo del proyecto deberá contemplar las siguientes actividades:

\begin{enumerate}
    \item Definición de la problemática a resolver.
    \item Definición de los casos de uso y modelos.
    \item Recopilación, especificación y selección de los requerimientos.
    \item Especificación técnica de la solución.
    \item Elaboración de prototipos.
\end{enumerate}

\subsection{Actividad 1}

En esta actividad los estudiantes deben establecer una problemática en un máximo de un párrafo.

Recuerde que la metodología de proyecto implica que:

\begin{enumerate}
    \item Deben abordar una problemática existente en el mundo empresarial o plantear una propuesta de valor asociada a un emprendimiento, que permita dar solución informática a ella. 
    \item El problema o situación puede ser entregado directamente por el académico o puede ser levantado por los estudiantes desde la realidad.
    \item El problema o situación que se pretende resolver, debe ser abordado por los estudiantes como un proyecto, que debe responder a la estructura que la profesión o disciplina disponga.
    \item Deberán  presentar información que permita justificar la problemática a abordar a través del proyecto. 
    \item Deberán formular los objetivos del proyecto (generales y específicos), los cuales tienen como propósito indicar los resultados que se pretenden alcanzar. 
    \begin{enumerate}
        \item Objetivo general: corresponde al resultado general que se pretende alcanzar con el proyecto. 
        \item Objetivos específicos: corresponden a las acciones progresivas que permitirán lograr el objetivo general. 
    \end{enumerate}
    \item Deberán establecer el contexto de producción del proyecto (grupo etario, sexo, lugar de aplicación, entre otros), de manera de poder identificar al grupo de usuarios de la solución. 
\end{enumerate}

\subsection{Actividad 2}

En esta actividad los estudiantes deben dar el contexto de las funcionalidades del proyecto y un modelo propuesto para la solución, para ello deberá considerar lo siguiente:

\begin{enumerate}
    \item Deberán definir 3 procesos de negocios relevantes. Recordar que un proceso de negocio es: <<Un conjunto de actividades y tareas que, una vez completadas, consiguen un objetivo prefijado
    para la empresa>>, por ende estos 3 procesos de negocios deberán estar enlazados a los objetivos específicos del proyecto.
    \item Deberán entregar un modelo BPMN que explique cada uno de los procesos de negocio a ser abordado.
    \item Por cada uno de los procesos de negocio deberán desarrollar el diagrama de casos de uso involucrado, identificando:
    \begin{enumerate}
        \item Casos de uso y sus relaciones (\textit{include}, \textit{extend} considerando los puntos de extensión, \textit{generalice}).
        \item Actores involucrados en el diagrama (tanto internos, como externos)\footnote{Recordar que los actores son quienes interactúan directamente con el sistema, si un actor no interactúa directamente 
        con el sistema, este no debería estar diagramado}.
    \end{enumerate}
    \item Por cada diagrama de casos de uso, deberán construir los casos de uso de alto nivel de todos los casos de uso involucrados en el diagrama.
    \item Deberán seleccionar 2 casos de uso por cada proceso (6 en total) y realizar:
    \begin{enumerate}
        \item Caso de uso extendido.
        \item Diagrama de clases.
        \item Diagrama de secuencia.
    \end{enumerate}
    \item Deberán presentar la propuesta de solución, modelando el esquema de datos a ser utilizado, para ello deberán proporcionar los siguientes diagramas:
    \begin{enumerate}
        \item Modelo conceptual de la solución (MER)\footnote{Modelo entidad relación}.
        \item Modelo lógico de la solución (MR)\footnote{Modelo relacional, se espera una normalización hasta 3FN o acorde a la arquitectura propuesta}.
    \end{enumerate}
\end{enumerate}

\subsection{Actividad 3}

En esta actividad deberán aplicar las técnicas aprendidas en ingeniería y gestión de requerimientos, cumpliendo como mínimo con:

\begin{enumerate}
    \item Obtener desde los casos de uso, los requerimientos candidatos.
    \item Elegir y priorizar los requerimientos a abordar en el desarrollo del proyecto.
    \item Documentar requerimientos según la norma IEEE830. (Anexo A)
\end{enumerate} 

Es importante en esta sección la identificación de los requerimientos mínimos necesarios para abordar el MVP de la solución y todo aquel conjunto de requerimientos que queden fuera de la propuesta de MVP, deberá
quedar detallado en las conclusiones del proyecto, indicando etapas futuras del proyecto y que requerimientos pudiesen ser abordados.

\subsection{Actividad 4}

En esta actividad se deberá detallar cuales son las especificaciones técnicas a considerar en el proyecto, así como aspectos de la metodología, siguiendo los lineamientos:

\begin{enumerate}
    \item Deberán definir el tipo de arquitectura de software a ser utilizada (MVC, MVVM, arquitectura hexagonal, arquitectura por microservicios o cualquier mixtura). Dando una breve explicación de la arquitectura a abordar
    y como será el planteamiento en la solución del software.
    \item Identificar el \textbf{stack} tecnológico sobre el cual trabajaran, detallando:
    \begin{enumerate}
        \item Lenguaje de programación a utilizar (tanto en frontend, como en backend), justificando la elección.
        \item De usar un framework de desarrollo, explicar que otras opciones existen (a lo menos 2) y cuál fue el motivo por el cual escogieron ese y no otro (esto aplica tanto para frameworks de \textit{frontend} como de \textit{backend} de la solución).
        \footnote{De no usar un framework, explicar las razones por las cuales no usarían un framework para desarrollar}
        \item Motor de base de datos a ser utilizado, explicando que otros motores podrían aplicar en la solución y justificando la razón por la cual escogen el motor señalado.
        \item Sistema operativo sobre el cual apuntan que esté orientada la solución servidor y sobre la cual sería necesario considerar la instalación del software (Linux / Windows), justificando la elección.
        \item De ser una aplicación orientada al desarrollo móvil, indicar:
        \begin{enumerate}
            \item Plataforma a la que apuntan.
            \item Tecnología a ser utilizada (código nativo / transpiladores a código nativo) justificando elección.
        \end{enumerate}
        \item Indicar el tipo de metodología que será utilizada por el equipo de trabajo, indicando:
        \begin{enumerate}
            \item Breve descripción de la metodología (no más de un párrafo).
            \item Detalle de como trabajarán con la metodología y sus principales limitantes a la hora de utilizar la metodología siguiendo la literatura.
            \item Explicación de la adaptaciones necesarias a ser consideradas en la metodología para poder ser utilizada en el proyecto.
        \end{enumerate}
    \end{enumerate}
\end{enumerate}

\subsection{Actividad 5}

En esta actividad deberán detallar los aspectos del prototipo a ser construido, utilizando para ello wireframes de baja fidelidad y explicando su flujo a través de diagramas de interacción y \textit{Storyboards}. Para ello deberán seguir los siguientes
lineamientos:

\begin{enumerate}
    \item Por cada proceso de negocio relevante deberán construir los \textit{wireframes} de baja fidelidad involucrados en el proceso.
    \item Estos \textit{wireframes} deberán contener una breve descripción de que requerimientos o casos de uso que consideran.
    \item Una vez explicado, deberán dar el diagrama de interacción entre estos \textit{wireframes} a través de un \textit{Storyboard} del proceso de negocio considerado.
    \item Deberán definir una guía de estilos a seguir en el desarrollo de la aplicación.
\end{enumerate}

\section{Forma de evaluación}

La forma de evaluación se encuentra detallada en la siguiente tabla:

\begin{longtable}[c]{|l|c|}
    \hline
    Detalle & Puntaje \\
    \hline
    \endfirsthead

    \hline
    Detalle & Puntaje \\
    \hline
    \endhead

    Actividad 1 & 10 puntos total \\
    \tabitem Definición del problema & 2 pts\\
    \tabitem Definición de objetivos & 3 pts\\
    \tabitem Definición de usuarios & 5 pts \\
    \hline
    Actividad 2 & 40 puntos total \\
    \tabitem Definición de los procesos de negocio & 3 pts \\
    \tabitem BPMN de los procesos de negocio & 3 pts (1 pts por cada uno) \\
    \tabitem Diagrama de casos de uso & 3 pts (1 pts por cada uno) \\
    \tabitem Casos de uso de alto nivel & 5 pts \\
    \tabitem Casos de uso extendido y diagramas & 6 pts \\
    \tabitem Modelo entidad relación & 10 pts \\
    \tabitem Modelo relacional & 10 pts \\
    \hline
    Actividad 3 & 10 puntos total \\
    \tabitem Listado de requerimientos funcionales & 4 pts \\
    \tabitem Listado de requerimientos no funcionales & 4 pts \\
    \tabitem Priorización de requerimientos de acuerdo a entregable & 2 pts \\ 
    \hline
    Actividad 4 & 10 puntos total \\
    \tabitem Definición de lenguaje de programación y tabla comparativa & 2 pts \\
    \tabitem Definición de framework y tabla comparativa & 2 pts \\
    \tabitem Definición de motor de base de datos y tabla comparativa & 2 pts \\
    \tabitem Definición de S.O a operar y justificación & 1 pts \\
    \tabitem Metodología de trabajo y adaptaciones propuestas & 3 pts \\
    \hline
    Actividad 5 & 30 puntos total \\
    \tabitem Wireframes por cada proceso de negocio & 10 pts \\
    \tabitem Desarrollo de \textit{Storyboard} con diagrama de interacción & 10 pts \\
    \tabitem Guía de estilos a utilizar & 10 pts \\
    \hline
\end{longtable}

\section{Potenciales descuentos}

Se podrán aplicar los siguientes descuentos al informe:

\begin{longtable}[c]{|l|c|}
    \hline
    Detalle & Puntaje \\
    \hline
    \endfirsthead

    \hline
    Detalle & Puntaje \\
    \hline
    \endhead

    Mala presentación del informe & 10 puntos \\
    \hline
    Omisión de índices de temas & 10 puntos \\
    \hline
    Mala introducción al informe & 10 puntos \\
    \hline
    Introducción inexistente & 15 puntos \\
    \hline
    Malas conclusiones o incompletas & 10 puntos \\
    \hline
    Omisión de conclusiones & 20 puntos \\
    \hline
    Faltas de ortografía y redacción & De 5 a 10 faltas - 5 puntos \\
    & De 11 a 20 faltas - 10 puntos \\
    & De 21 a 25 faltas - 15 puntos \\
    & Más de 25 faltas - 20 puntos \\
    \hline
\end{longtable}

\section{Fecha de entrega}

El trabajo deberá ser entregado por el jefe de proyecto a través del AAI antes del día 30 de abril del 2021, y tendrán hasta las 23:55 hrs de ese mismo día para subir el archivo en formato PDF o Word.

No se recibirán trabajos fuera de ese horario o a través de otro medio que no sea el AAI.

\end{document}